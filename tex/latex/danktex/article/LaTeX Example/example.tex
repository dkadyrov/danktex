\documentclass{web_article}

% \documentclass{article}
% \usepackage[letterpaper,margin=1in]{geometry}

% \usepackage{lipsum}
% Remove Paragraph Indents
% \setlength{\parindent}{0pt}
% \setlength{\parskip}{6pt plus 2pt minus 1pt}

% TODO Black and White
% TODO Finish Markdown

% Colors
% \usepackage[dvipsnames]{xcolor}

% Get Markdown Variables
\title{Pandoc Article Class}
\author{Daniel Kadyrov}
\date{\today}

% Headers and Footer
% \usepackage{fancyhdr}
% \pagestyle{fancy}
% \fancyhf{}
% \lhead{\title}
% \chead{}
% \rhead{\leftmark}
% \rfoot{Page \thepage}
% \lfoot{}
% \renewcommand{\headrulewidth}{1pt}
% \renewcommand{\footrulewidth}{1pt}

% % Hyperref
% \usepackage{hyperref}
% \hypersetup{
%   %   %   %   colorlinks,%
%   citecolor=black,%
%   filecolor=black,%
%   linkcolor=black,%
%   urlcolor=black,
%   breaklinks=true}
% \urlstyle{same}  % don't use monospace font for urls

% Quotations
% Define Quote Font
% % Graphics / Images

% % \usepackage{graphicx,grffile}
% \usepackage{float}
% \makeatletter
% \def\maxwidth{\ifdim\Gin@nat@width>\linewidth\linewidth\else\Gin@nat@width\fi}
% \def\maxheight{\ifdim\Gin@nat@height>\textheight\textheight\else\Gin@nat@height\fi}
% \makeatother
% % Scale images if necessary, so that they will not overflow the page
% % margins by default, and it is still possible to overwrite the defaults
% % using explicit options in \includegraphics[width, height, ...]{}
% \setkeys{Gin}{width=\maxwidth,height=\maxheight,keepaspectratio}
% \let\origfigure=\figure
% \let\endorigfigure=\endfigure
% \renewenvironment{figure}[1][]{%
%    \origfigure[H]
% }{%
%    \endorigfigure
% }
% 
% % \usepackage{listings}
% \newcommand{\passthrough}[1]{#1}
% % 
%
% listing colors
%
% \definecolor{listing-background}{HTML}{F7F7F7}
% \definecolor{listing-rule}{HTML}{B3B2B3}
% \definecolor{listing-numbers}{HTML}{B3B2B3}
% \definecolor{listing-text-color}{HTML}{000000}
% \definecolor{listing-keyword}{HTML}{435489}
% \definecolor{listing-identifier}{HTML}{435489}
% \definecolor{listing-string}{HTML}{00999A}
% \definecolor{listing-comment}{HTML}{8E8E8E}
% \definecolor{listing-javadoc-comment}{HTML}{006CA9}

% Listings
%
%

% % \lstdefinestyle{eisvogel_listing_style}{
%   language         = java,
%   numbers          = left,
%   backgroundcolor  = \color{listing-background},
%   basicstyle       = \color{listing-text-color}\small\ttfamily{}\linespread{1.15}, % print whole listing small
%   xleftmargin=.075\textwidth,
%   xrightmargin=.01\textwidth,
%   breaklines       = true,
%   frame            = single,
%   framesep         = 10 pt,
%   rulecolor        = \color{listing-rule},
%   frameround       = ffff,
%   % framexleftmargin = 2.5em,
%   tabsize          = 4,
%   numberstyle      = \color{listing-numbers},
%   numbersep        = 20pt,
%   aboveskip        = 1.0em,
%   keywordstyle     = \bfseries\color{listing-keyword},
%   classoffset      = 0,
%   sensitive        = true,
%   identifierstyle  = \color{listing-identifier},
%   commentstyle     = \color{listing-comment},
%   morecomment      = [s][\color{listing-javadoc-comment}]{/**}{*/},
%   stringstyle      = \color{listing-string},
%   showstringspaces = false,
%   escapeinside     = {/*@}{@*/}, % Allow LaTeX inside these special comments
%   literate         =
%   {á}{{\'a}}1 {é}{{\'e}}1 {í}{{\'i}}1 {ó}{{\'o}}1 {ú}{{\'u}}1
%   {Á}{{\'A}}1 {É}{{\'E}}1 {Í}{{\'I}}1 {Ó}{{\'O}}1 {Ú}{{\'U}}1
%   {à}{{\`a}}1 {è}{{\'e}}1 {ì}{{\`i}}1 {ò}{{\`o}}1 {ù}{{\`u}}1
%   {À}{{\`A}}1 {È}{{\'E}}1 {Ì}{{\`I}}1 {Ò}{{\`O}}1 {Ù}{{\`U}}1
%   {ä}{{\"a}}1 {ë}{{\"e}}1 {ï}{{\"i}}1 {ö}{{\"o}}1 {ü}{{\"u}}1
%   {Ä}{{\"A}}1 {Ë}{{\"E}}1 {Ï}{{\"I}}1 {Ö}{{\"O}}1 {Ü}{{\"U}}1
%   {â}{{\^a}}1 {ê}{{\^e}}1 {î}{{\^i}}1 {ô}{{\^o}}1 {û}{{\^u}}1
%   {Â}{{\^A}}1 {Ê}{{\^E}}1 {Î}{{\^I}}1 {Ô}{{\^O}}1 {Û}{{\^U}}1
%   {œ}{{\oe}}1 {Œ}{{\OE}}1 {æ}{{\ae}}1 {Æ}{{\AE}}1 {ß}{{\ss}}1
%   {ç}{{\c c}}1 {Ç}{{\c C}}1 {ø}{{\o}}1 {å}{{\r a}}1 {Å}{{\r A}}1
%   {€}{{\EUR}}1 {£}{{\pounds}}1 {«}{{\guillemotleft}}1
%   {»}{{\guillemotright}}1 {ñ}{{\~n}}1 {Ñ}{{\~N}}1 {¿}{{?`}}1
%   {…}{{\ldots}}1 {≥}{{>=}}1 {≤}{{<=}}1 {„}{{\glqq}}1 {“}{{\grqq}}1
%   {”}{{''}}1
% }
% \lstset{style=eisvogel_listing_style}
%
% % \lstdefinelanguage{XML}{
% %   morestring      = [b]",
% %   moredelim       = [s][\bfseries\color{listing-keyword}]{<}{\ },
% %   moredelim       = [s][\bfseries\color{listing-keyword}]{</}{>},
% %   moredelim       = [l][\bfseries\color{listing-keyword}]{/>},
% %   moredelim       = [l][\bfseries\color{listing-keyword}]{>},
% %   morecomment     = [s]{<?}{?>},
% %   morecomment     = [s]{<!--}{-->},
% %   commentstyle    = \color{listing-comment},
% %   stringstyle     = \color{listing-string},
% %   identifierstyle = \color{listing-identifier}
% % }
% 
%--------------------------------------------------------------


\begin{document}


\maketitle


\section{Using this Script}\label{using-this-script}

\subsection{What is Pandoc?}\label{what-is-pandoc}

\subsection{Using Custom Templates}\label{using-custom-templates}

\newpage

\section{Markdown Syntax}\label{markdown-syntax}

\subsection{Text Operations}\label{text-operations}

The basic markdown text operations can be converted into LaTeX, Word,
and into a PDF through Pandoc.

\begin{lstlisting}
Words can set in **bold** or they can be set in *italics*
\end{lstlisting}

Words can set in \textbf{bold} or they can be set in \emph{italics}

\subsection{Hyperlinks}\label{hyperlinks}

\begin{lstlisting}
Enclosed Links: <http://google.com>

Inline links: [one with a title](http://fsf.org "click here for a good time!")

Emails: [Write me!](mailto:sam@green.eggs.ham)

Reference Links: [Sends to Section 2.1.1 Hyperlinks](#Hyperlinks)
\end{lstlisting}

Enclosed Links: \url{http://google.com}

Inline links: \href{http://fsf.org}{one with a title}

Emails: \href{mailto:sam@green.eggs.ham}{Write me!}

Reference Links: \protect\hyperlink{Hyperlinks}{Sends to Section 2.1.1
Hyperlinks}

\subsection{Quotes}\label{quotes}

Quotes are the highlight Markdown. They are achieved in LaTeX through
the mdframed package.

\begin{quote}
This is a block quote.

\begin{quote}
A block quote within a block quote.
\end{quote}
\end{quote}

\newpage

\subsection{Figures}\label{figures}

\begin{figure}
\centering
\includegraphics{300.png}
\caption{Image 1}
\end{figure}

\newpage

\subsection{Footnotes}\label{footnotes}

Here is a footnote reference,\footnote{Here is the footnote.} and
another.\footnote{Here's one with multiple blocks.}

Subsequent paragraphs are indented to show that they belong to the
previous footnote.

\subsubsection{Inline Footnotes}\label{inline-footnotes}

Here is an inline note.\footnote{Inlines notes are easier to write,
  since you don't have to pick an identifier and move down to type the
  note.}

\subsection{Code Inserts}\label{code-inserts}

\lipsum[2-4]

\subsubsection{Verbatim}\label{verbatim}

\newpage

\subsubsection{Syntax Highlighted}\label{syntax-highlighted}

\begin{lstlisting}[language=Python]
# Example of a Comments
class MyClass(Yourclass):
    def __init__(self, my, yours):
        print('This is a string')
\end{lstlisting}

\newpage

\subsection{\texorpdfstring{\LaTeX}{}}\label{section}

Traditional \LaTeX can be inserted by using the backward slash command.

\subsubsection{Mathematical Equations}\label{mathematical-equations}

Mathematical equations can be inserted in-line such as \(F = ma\) by
using the \$ between expressions. For equations however, the
begin\{equation\} enviroment needs to be used, such as the example
below:

\begin{equation}
F = ma 
\end{equation}

\end{document}
